% !TEX root = ./document.tex

\documentclass{article}

\usepackage{mystyle}
\usepackage{myvars}

%-----------------------------

\begin{document}

  \maketitle

  %-----------------------------
  %  TEXT
  %-----------------------------

  \section{Introducción}

    \paragraph{}
    [TODO ]

    \paragraph{}
    Denotaremos por $U = U_1 \cup ... \cup U_h \cup... \cup U_L = \{1, ...,k,...,N\} $ a la población, para la cual tenemos una división en $L$ estratos denotando por $U_h$ al estrato $h \in \{1,..., L\}$. Sea $I_h$ el conjunto de índices de las observaciones seleccionadas en el estrato $U_h$ y $s_h$ la muestra extraida de dicho estrato. Por tanto, podemos denotar a la muestra global por $s = s_1 \cup ... \cup s_h \cup ... \cup s_L$

    \paragraph{}
    En este caso, tal y como se ha indicado anteriormente se va a presuponer la utilización del método \emph{m.a.s.} sobre cada estrato, caracterizado porque el tamaño de la muestra se fija \emph{a-priori} y se seleccionan las observaciones \emph{sin reemplazamiento}, es decir, una vez seleccionada una observación, esta desaparece del conjunto de candidatos a aparecer en la muestra. Por tanto, no hay observaciones repetidas en la muestra.

    \paragraph{}
    Denotaremos por $\Tau$ al total poblacional de una determinada variable de interés $Y$ denotando como $y_k \quad \forall k \in U$ al \emph{$k$-ésimo} valor de $Y$. Es fácil entender por tanto que el total poblacional se define como $\Tau = \sum\limits_U y_k$.

    \paragraph{}
    Denotaremos por $P$ a la proporción poblacional de una determinada variable de interés $Y$ de carácter binario denotando como $y_k \quad \forall k \in U$ al \emph{$k$-ésimo} valor de $Y$. Es fácil entender por tanto que la proporción poblacional se define como $P = \frac{\sum\limits_U y_k}{N}$.

    \paragraph{}
    Bajo la hipótesis de \emph{muestreo aleatorio simple (m.a.s.)}, a partir del cual se pretende obtener una aproximación lo más precisa posible tanto del total poblacional $\Tau$ como de la proporción poblacional $P$. Para ello, es necesario apoyarse en los valores del tamaño de la población $N$, el del estrato $U_h$ como $N_h$ y el de la muestra $s_h$ denotado por $n_h$. El tamaño relativo del estrato se define como $W_h = \frac{N_h}{N}$. También se define el tamaño relativo de la muestra como $f_h = \frac{n_h}{N_h}$. Entonces, en este caso un buen estimador del total poblacional es el $\pi$-estimador $\widehat{\Tau} = \sum\limits_{s_h} \frac{y_k}{W_h}$. Para el caso del estimador de la proporción poblacional, un buen estimador es $\widehat{P} = \frac{\sum\limits_{s_h} \frac{y_k}{W_h}}{N}$

    \paragraph{}
    La varianza del estimador del total sobre cada estrato se define a continuación:

    \begin{align}
      Var(\widehat{\Tau_h}) &= \\
      &=\frac{N_h^2 (1-f_h)\sigma_h^{2*}}{n_h} \\
      &= \frac{N_h^2 (1-\frac{n_h}{N_h})\sigma_h^{2*}}{n_h} \\
      &= \frac{(N_h^2- N_hn_h)\sigma_h^{2*}}{n_h} \\
      &= \left(\frac{N_h^2}{n_h}- N_h\right)\sigma_h^{2*}
    \end{align}

    \paragraph{}
    La varianza del estimador de la proporción sobre cada estrato se define a continuación:

    \begin{align}
      Var(\widehat{P_h}) &= \\
      &= ... \\
      &= \frac{N_h-n_h}{n_h*(N_h-1)}\widehat{P_h} * (1-\widehat{P_h})
    \end{align}


    \begin{align}
      w_h = \frac{n_h}{n}
    \end{align}

    \begin{align}
      \frac{N_h}{N_h -1} \simeq 1
    \end{align}

    \begin{align}
      B^2 = k^2 \sum\limits_{h=1}^L W_h^2 \frac{(1-f_h)}{n_h}\sigma_h^{*2}
    \end{align}

    \begin{align}
      n_h = n*w_h
    \end{align}

    \begin{align}
      w_h &= W_h &\text{(afijación proporcional)}\\
      w_h &= \frac{N_h\sigma_h^*}{\sum\limits_{i=1}^LN_i\sigma_i^*}&\text{(afijación mínima varianza)}
    \end{align}

    \begin{align}
      B^2 &= k^2 \sum\limits_{h=1}^L W_h^2 \frac{(1-f_h)}{n_h}\sigma_h^{*2} \\
      \frac{B^2}{k^2} &=  \sum\limits_{h=1}^L W_h^2 \frac{(1-f_h)}{n*w_h}\sigma_h^{*2} \\
      \frac{B^2}{k^2} &=  \frac{\sum\limits_{h=1}^L W_h^2 \frac{(1-f_h)}{w_h}\sigma_h^{*2}}{n} \\
      n &=  \frac{\sum\limits_{h=1}^L W_h^2 \frac{(1-f_h)}{w_h}\sigma_h^{*2}}{\frac{B^2}{k^2}} \\
      n &= \frac{\sum\limits_{h=1}^L\frac{W_h^2}{w_h}\sigma_h^{*2}}{\frac{B^2}{k^2}+\sum\limits_{h=1}^L\frac{W_h^2}{N_h}\sigma_h^{*2}}
    \end{align}

    \paragraph{}
    Una vez desarrollada la ecuación que a partir de la cual determinar el tamaño de muestra global $n$, fijado un erro de estimación $b$ a una confianza de nivel $k$, el siguiente paso es particularizar esto para estimadores concretos. Esto consiste simplemente en la substitución del valor de la varianza por el del estimador en cuestión. En la sección \ref{sec:dem1} se realiza dicha demostración para el caso del total poblacional $\widehat{\Tau_h}$ y en la sección \ref{sec:dem2} se realiza para el caso del estimador de proporción poblacional $\widehat{P_h}$.


  \section{Demostración para estimador del total poblacional $\widehat{\Tau_h}$}
  \label{sec:dem1}
    \paragraph{}
    [TODO ]

    \begin{align}
      n &= \\
      &= \frac{\sum\limits_{h=1}^L\frac{W_h^2}{w_h}\sigma_h^{*2}}{\frac{B^2}{k^2}+\sum\limits_{h=1}^L\frac{W_h^2}{N_h}\sigma_h^{*2}} \\
      &= ... \\
      &= \frac{\sum\limits_{h=1}^{L} \frac{N_h^2}{w_h}\sigma_h^{*2}}
              {\frac{B^2}{K^2}+\sum\limits_{h=1}^{L}\sigma_h^* N_h}
    \end{align}

  \section{Demostración para estimador de proporción poblacional $\widehat{P_h}$}
  \label{sec:dem2}

    \paragraph{}
    [TODO ]

    \begin{align}
      n &= \\
      &= \frac{\sum\limits_{h=1}^L\frac{W_h^2}{w_h}\sigma_h^{*2}}{\frac{B^2}{k^2}+\sum\limits_{h=1}^L\frac{W_h^2}{N_h}\sigma_h^{*2}} \\
      &= ... \\
      &= \frac{\sum\limits_{h=1}^L \frac{W_h^2}{w_h}\frac{N_h}{N_h-1}P_h(1-P_h)}
              {\frac{B^2}{K^2} + \sum\limits_{h=1}^LP_h(1-P_h)\frac{W_h}{N}\frac{N_h}{N_h-1}}
    \end{align}



  %-----------------------------
  %  Bibliographic references
  %-----------------------------

  \nocite{muest2017}
  \nocite{sarndal2003model}

  \bibliographystyle{acm}
  \bibliography{bib}

\end{document}
